The increasing demand for voice and data capacity has been the primary motivation for cellular and PCS evolution.
Cohere detection is the most power-efficient scheme that is capable of providing substantial improvement in system capacity over noncoherent and differentially coherent schemes.
For this reason, reverse link coherent detection is being considered as the framework for third generation wireless communication systems.
In mobile communications, however. rapid fading may preclude a good estimate of the channel phase required to achieve coherent demodulation.
This may lead to serious degradation to system performance. 
This dissertation investigates the capacity and error-rate performance of coherent systems with imperfect carrier recovery.
These systems are known as partially coherent systems.\\

Partially coherent systems have not received thorough investigation in the literature.
Most of the previous work has been focused on the analysis of performance for BPSK over AWGN channels.
Upper bounds on bit error probability have been derived, but found to be very conservative for the range of carrier phase jitter variance of practical interest.
The error performance for partially coherent QPSK has not received much attention.
Furthermore, the performance of partially coherent systems over multipath fading channels with diversity has not been studied.\\

In this dissertation, several upper and lower bounds on the error performance of partially coherent systems are derived by the application of Jensen's inequality and the isomorphism theorem from the theory of moment spaces assuming that the carrier phase error could have either Tikhonov or Gaussian distribution.
An analytical method based on Gram-Charlier series expansion is also developed for the computation of the error probability and signal-to-noise ratio distribution of partially coherent systems over fading channels with diversity.\\

The application of partially cohere systems for CDMA mobile cellular communication is also investigated.
performance impairments due to thermal noise, multipath fading, multiple access interference and self-noise are included in the analysis.
A design criterion for adding weak signals with equal gain combining is established when the multipath intensity profile is nonuniform.